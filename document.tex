\documentclass[a4paper, 11pt]{article}
\usepackage[fleqn]{amsmath}
\usepackage[ngerman]{babel}
\usepackage{paralist}
\usepackage[utf8]{inputenc}
\usepackage{fullpage}
\author{Jannis Hübl & Tobias Bechtold}
\title{Zusammenfassung Statistik}
\date{November 2013}

\begin{document}
\maketitle
\newpage
\tableofcontents

\section{Descriptive Statistik}
\section{Bivarite Stichproben}
\subsection{lineare Korelationskoeffizent}
\subsection{Spearmensche Korelationskoeffizent}
\section{Bedingte Warscheinlichkeit}
\section{Varianz und Erwartungswert}
\section{Verteilungen}
\subsection{Binomialverteilung}
Warscheinlichkeit:
\newline $P(x=k) = \binom{n}{k} p^k (1-p)^{n-k}$
\newline\newline Warscheinlichkeitsdichte:
\newline $f(x) = \binom{n}{x} p^x(1-p)^{n-x}$
\newline\newline Erwartungswert:
\newline $\mu = E(x) = n * p$
\newline\newline Standartabweichung:
Wo kommt das q her?
\newline $\sigma^2 = V(x)= n * p * q$
\subsection{Poissionverteilung}
Mittelwert:
\newline $\mu = n * p$
\newline\newline Warscheinlichkeit:
\newline $P(x=k) = \frac{\mu^k} {k!} e^{-\mu}$
\newline\newline Warscheinlichkeitsdichte:
\newline $f(x) = \frac{\mu^x} {x!} e^{-\mu}$
\newline\newline Erwartungswert:
\newline $\mu = E(x) = n * p$
\newline\newline Standartabweichung:
\newline $\sigma^2 = V(x)= n * p = \mu $
\subsection{Gleichverteilung}
Warscheinlichkeitsdichte:
\newline $f(x) = \begin {cases}
  \frac{1} {b-a}, & \text {für } a\le x \le b \\
  0, & \text{sonst}
\end{cases} $
\newline\newline Erwartungswert:
\newline $ \mu = E(x) = \frac { a + b } {2}$
\newline\newline Standartabweichung:
\newline $\sigma^2 = V(x)= \frac {(a + b)^2 } {12} $
\subsection{Normalverteilung}
\subsection{$\chi^2-Verteilung$}
\subsection{T-Verteilung}
\section{Carakteristische Funktion}
\section{Maximum-Likelihood-Methode (LM)}
\section{Konvidenz intervalle}
\section{Tests}
\subsection{T-Test}
\subsection{Biomial-Test}
\subsection{$\chi^2$-Test}

\end{document}
