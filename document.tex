\documentclass[a4paper, 11pt]{article}
\usepackage[fleqn]{amsmath}
\usepackage[ngerman]{babel}
\usepackage{paralist}
\usepackage[utf8]{inputenc}
\usepackage{fullpage}
\author{Jannis Hübl & Tobias Bechtold}
\title{Zusammenfassung Statistik}
\date{November 2013}

\begin{document}
\maketitle
\newpage
\tableofcontents

\section{Descriptive Statistik}
Absolute Häufigkeit: Anzahl der Ausprägungen in einem Intervall. \\
Bsp.: $x_1=28, x_2=30, x_3=30, x_4=32$ \\
\newline
Diese Ausprägungen teilen wir nun in folgende Intervalle ein: \\
$\Delta{x_1} = [0, 29], \Delta{x_2} = [30, 31], \Delta{x_3} = [32, \infty]$ \\
\newline
Nun sehen wir die absolute Häufigkeit der Merkmale in $\Delta{x_1} = h_1 = 1,
h_2 = 2, h_3 = 1$ \\
\newline
Die relative Häufigkeit gibt nun die absolute Häufigkeit relativ zur Anzahl an
Merkmalen an, also $h_1^{rel} = \frac{1}{4} = 0,25, h_2^{rel} = \frac{2}{4} =
0,5, h_3^{rel} = \frac{1}{4} = 0,25$. \\
\newline
Kumulierte Häufigkeit (Summe aller absoluten Häufigkeiten):
$S_i=\sum\nolimits_{j=1}^{i} h_j$.\\
relative kumulierte Häufigkeit: $S_i^{rel}=\sum\nolimits_{j=1}^{i} h_j^{rel}$.\\
\newline
Arithmetisches Mittel (Mittelwert): $\bar{x} = \frac{1}{n}
\sum\nolimits_{j=1}^{n} x_j$\\
\newline
Gewichtetes arithmetisches Mittel: einzelne Werte werden mit der jeweiligen
Gruppengröße $w_j$ gewichtet: $\bar{x}=\frac{1}{W}\sum\nolimits_{j=1}^{n}w_j
x_j$ mit $W=\sum\nolimits_{j=1}^{n} w_j$. Siehe Bsp. 2.9 auf S. 14
\newline
\section{Bivarite Stichproben}
\subsection{lineare Korelationskoeffizent}
\subsection{Spearmensche Korelationskoeffizent}
\section{Bedingte Warscheinlichkeit}
\section{Varianz und Erwartungswert}
\section{Verteilungen}
\subsection{Binomialverteilung}
Warscheinlichkeit:

\newline $P(x=k) = \binom{n}{k} p^k (1-p)^{n-k}$
\newline\newline Warscheinlichkeitsdichte:

$f(x) = \binom{n}{x} p^x(1-p)^{n-x}$
\newline\newline Erwartungswert:

$\mu = E(x) = n * p$
\newline\newline Standartabweichung:
Wo kommt das q her?

$\sigma^2 = V(x)= n * p * q$
\subsection{Poissionverteilung}
Mittelwert:

$\mu = n * p$
\newline\newline Warscheinlichkeit:

$P(x=k) = \frac{\mu^k} {k!} e^{-\mu}$
\newline\newline Warscheinlichkeitsdichte:

$f(x) = \frac{\mu^x} {x!} e^{-\mu}$
\newline\newline Erwartungswert:

$\mu = E(x) = n * p$
\newline\newline Standartabweichung:

$\sigma^2 = V(x)= n * p = \mu $
\subsection{Gleichverteilung}
Warscheinlichkeitsdichte:

$f(x) = \begin {cases}
  \frac{1} {b-a}, & \text {für } a\le x \le b \\
  0, & \text{sonst}
\end{cases} $
\newline\newline Erwartungswert:

$ \mu = E(x) = \frac { a + b } {2}$
\newline\newline Standartabweichung:

$\sigma^2 = V(x)= \frac {(a + b)^2 } {12} $
\subsection{Normalverteilung}

\subsubsection{Allgemein}
Warscheinlichkeitsdichte: 

$f(x) = \frac{1} {\sigma \sqrt{2\pi}} e^{(-\frac{1}{2}(\frac{x-\mu}{\sigma})^2)}$
\newline\newline Erwartungswert:
\newline Der Erwartugswert kann durch ableiten errechnet werden.

$f'(x) = -\frac{x-\mu} {\sigma^2} f(x)$
$\mu$ erhält man wenn man f'(x) = 0 setzt.
\newline\newline Standartabweichung:
\newline Die Standartabweichung erhält man durch 2 maliges ableiten.


$f''(x) = \frac{1}{\sigma^2} (\frac{1}{\sigma^2} (x-\mu)^2 - 1) f(x)$


$f''(x) = 0 $ liefert $ x = \mu \pm \sigma$

\subsubsection{Standardnormalverteilung}
Für die Standardnormalverteilung gilt:


$\mu = 0$


$\sigma = \pm 1$

\subsection{$\chi^2-Verteilung$}
\subsection{T-Verteilung}
\section{Carakteristische Funktion}
\section{Maximum-Likelihood-Methode (LM)}
\section{Konvidenz intervalle}
\section{Tests}
\subsection{T-Test}
Bei einer Befragung gaben 10 Studenten an: 26,26,24,28,20,30,22,28 SWS (Semester Wochenstunden) zu belegen.

$\mu = \frac {1} {n} \sum x_i = 25$
\begin{enumerate}
  \item Hypotese aufstellen:\\
     Man vesucht diese Hypotese zu wiederlegen $=>$ Die Hypotese muss das Gegenteil sein, was man beweisen will.
     \begin {itemize}
       \item Ein Mittelwert soll größer als ein betimmter Wert sein:\\
         Prüfen Sie das Studenten im Mittel mehr als 21.1 Semester Wochenstunden belegen.\\
         $=> H_0 = \mu \leq 21.1$
       \item Ein Mittelwert soll kleiner als ein bestimmer Wert sein:\\
         Prüfen Sie das Studenten im Mittel weniger als 21.1 SWS belegen\\
         $=> H_0 = \mu \geq 21.1$
       \item Ein Mittelwert soll nicht Signifikant von einem Wert abweichen.\\
         Prüfen Sie das Studenten im Mittel 21.1 SWS belegen.\\
         $=> H_0 = \mu = 21.1$
     \end{itemize}

   \item Erwarteter Wert ausrechen: \\
     \begin {itemize}
       \item
         Wenn die Varianz gegeben ist ist die Variable Standartnormalverteilt($N(0,1)$)
       \item 
         Wenn man die Varianz selber aus Daten berechent ist die Variable T-verteilt.\\
         $s^2 = \frac 1 {n-1} \sum_{i = 0} ^ n (x_i - \mu) ^2$\\
       $s_n^2 = \frac {s^2} n$

         Im Bsp: 

         $s^2 = \frac 1 {n-1} (26 - 25)^2 +(26 - 25)^2 + (24 - 25)^2 + ..  = $\\
         $      \frac 1 {9} 90 = 10$\\
         $s_n^2 = \frac {10} {10} = 1$

     \end {itemize}

     $ t_{expect} = \frac{\bar{x} - \mu} { \sqrt{s_n^2}}$

     Im Bsp:

     $ t_{expect} = \frac{ 25 - 21.1} { \sqrt{1^2}} = 3.9$
\end{enumerate}
\subsection{Biomial-Test}
\subsection{$\chi^2$-Test}

\end{document}
