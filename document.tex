\documentclass[a4paper, 11pt]{article}
\usepackage[fleqn]{amsmath}
\usepackage[ngerman]{babel}
\usepackage{paralist}
\usepackage[utf8]{inputenc}
\usepackage{fullpage}
\title{Zusammenfassung Statistik}
\author{Jannis Hübl & Tobias Bechtold}
\date{November 2013}

\begin{document}
\maketitle
\newpage
\tableofcontents
\newpage
\section{Descriptive Statistik}
\subsection{Absolute Häufigkeit}
Anzahl der Ausprägungen in einem Intervall.\\
Bsp.: $x_1=28, x_2=30, x_3=30, x_4=32$ \\
\newline
Diese Ausprägungen teilen wir nun in folgende Intervalle ein: \\
$\Delta{x_1} = [0, 29], \Delta{x_2} = [30, 31], \Delta{x_3} = [32, \infty]$ \\
\newline
Nun sehen wir die absolute Häufigkeit der Merkmale in $\Delta{x_1} = h_1 = 1,
h_2 = 2, h_3 = 1$ \\
\newline
Die relative Häufigkeit gibt nun die absolute Häufigkeit relativ zur Anzahl an
Merkmalen an, also $h_1^{rel} = \frac{1}{4} = 0,25, h_2^{rel} = \frac{2}{4} =
0,5, h_3^{rel} = \frac{1}{4} = 0,25$. \\
\newline
Kumulierte Häufigkeit (Summe aller absoluten Häufigkeiten):
$S_i=\sum\nolimits_{j=1}^{i} h_j$.\\
relative kumulierte Häufigkeit: $S_i^{rel}=\sum\nolimits_{j=1}^{i} h_j^{rel}$.\\
\subsection{Lagemaße}
Arithmetisches Mittel (Mittelwert): $\bar{x} = \frac{1}{n}
\sum\nolimits_{j=1}^{n} x_j$\\
\newline
Gewichtetes arithmetisches Mittel: einzelne Werte werden mit der jeweiligen
Gruppengröße $w_j$ gewichtet: $\bar{x}=\frac{1}{W}\sum\nolimits_{j=1}^{n}w_j
x_j$ mit $W=\sum\nolimits_{j=1}^{n} w_j$. Siehe Bsp. 2.9 auf S. 14 \\
\newline
Geometrisches Mittel:
$\bar{x}_{geo}=(\prod\nolimits_{j=1}^{n}x_j)^{\frac{1}{n}}$\\
\subsection{Median} Der Wert der in einer sortierten Reihe aller Daten genau in
der Mitte liegt. \\
Falls n ungerade: $\bar{x}_{med} = x_{\frac{n+1}{2}}$ \\
Falls n gerade: $\bar{x}_{med} = \frac{1}{2} \cdot (x_{\frac{n}{2}} +
x_{(\frac{n}{2} + 1)})$ \\ 
\subsection{Quantile}
Das p-Quantil $\bar{x}_p$ einer Stichprobe ist der Wert, unterhalb dessen ein
Anteil p aller Beobachtungen liegt. \\
Oft verwendete Spezialfälle: \\
Erstes Quartil $Q1 = x_i$ für $F(x_i) = 0.25$ \\
Zweites Quartil $Q2 = x_i$ für $F(x_i) = 0.5$ \\
Drittes Quartil $Q3 = x_i$ für $F(x_i) = 0.75$ \\
\newline
Für jedes p wird folgendermaßen vorgegangen: \\
Falls $n\cdot p$ keine ganze Zahl: $\bar{x}_p = x_k$ wobei $k$ die
unmittelbar auf $n\cdot p$ folgende ganze Zahl ist. \\
Falls $n\cdot p$ ganzzahlig: $\bar{x}_p = \frac{1}{2}(x_k+x_{k+1})$ mit $k =
n\cdot p$ \\
\subsection{Streuungsmaße}
\subsubsection{Variationsbreite}
$r=max_ix_i - min_ix_i$
\subsubsection{Interquartilsabstand}
Die Differenz der Quartile 3 und 1 $I = Q3-Q1$
\subsubsection{Quantilsabstände}
$x_{1-p}-x_p$
\subsubsection{Varianz}
$s^2 = \frac{1}{n-1} \sum\nolimits_{i=1}^{n}(x_i-\bar{x})^2$
\subsubsection{Standardabweichung}
$s=\sqrt{s^2}$
\subsubsection{Variationskoeffizient}
$v=\frac{s}{\bar{x}}$
\section{Bivariate Stichproben}
\subsection{Kovarianz}
Die empirische Kovarianz $s_{xy}$ ist ein Maß für den Zusammenhang
zweier streuender Variablen $x$ und $y$. \\
\newline
$s_{x,y}=\frac{\sum\nolimits_{i=1}^{n}(x_i - \bar{x})(y_i - \bar{y})}{n-1}$
\subsection{lineare Korelationskoeffizent}
$r_{xy} = \frac{s_{xy}}{s_{x}\cdot s_{y}} =
\frac{\sum\nolimits_{i=1}^{n}(x_i-\bar{x})(y_i-\bar{y})}{\sqrt{\sum\nolimits_{i=1}^{n}(x_i-\bar{x})^2
\cdot \sum\nolimits_{i=1}^{n}(y_i-\bar{y})^2}}$
\subsection{Spearmanscher Rangkorrelationskoeffizent}
$r_s = 1-\frac{6\sum\nolimits_{i=1}^{n}(r_{x,i}-r_{y,i})^2}{n(n^2-1)}$ \\
\newline
Beispiel: Zwei Messreihen $x$ und $y$: \\
\begin{tabular}{l r}
$(0,1,2,3)_x $ & $(8,5,0,1)_y $ \\
\end{tabular} \\
Anhand der Reihenfolge legen wir die Rangzahlen $r_{x,i}$ und $r_{y,i}$
folgendermaßen fest: \\
\begin{tabular}{l r}
$(1,2,3,4)_x $ & $(4,3,1,2)_y $ \\
\end{tabular} \\
Jetzt setzen wir die Rangzahlen in die Formel ein: \\
$r_s = 1-\frac{6(3^2+1^2+2^2+2^2)}{4(4^2-1)} = -0.8$
\newpage
\section{Bedingte Warscheinlichkeit}
Bedingte Wahrscheinlichkeit ist das Eintreten eines Ereignisses A unter der
Vorraussetzung, dass das Ereignis B bereits vorher eingetreten ist. \\
Beispiel: Mit welcher Wahrscheinlichkeit ziehe ich aus nem Kartenspiel eine 7
wenn vorher eine Zahl gezogen wurde? \\
Es wird geschrieben $P(\mathcal{A}|\mathcal{B})$ (lies: P von $\mathcal{A}$
unter der Voraussetzung $\mathcal{B}$). \\
Beispiel: Wie hoch ist die Wahrscheinlichkeit eine Augenzahl kleiner als 4 zu
würfeln, wenn bekannt ist, dass eine ungerade Augenzahl gewürfelt worden ist? \\
Es ist $P(\mathcal{A}|\mathcal{B})$ zu berechnen mit $\mathcal{A} = \{1,2,3\}$,
$P(\mathcal{A})= 0.5$ und $\mathcal{B} = \{ 1,3,5\},
P(\mathcal{B})=0.5$.
Es gilt $P(\mathcal{A}\cap \mathcal{B}) = P(\{ 1,3\}) =
\frac{1}{3}$, und ergibt sich $P(\mathcal{A}|\mathcal{B})= \frac{2}{3}$.
\section{Varianz und Erwartungswert}
\section{Verteilungen}
\subsection{Binomialverteilung}
Warscheinlichkeit:\\
$P(x=k) = \binom{n}{k} p^k (1-p)^{n-k}$ \\
Warscheinlichkeitsdichte:

$f(x) = \binom{n}{x} p^x(1-p)^{n-x}$
\newline\newline Erwartungswert:

$\mu = E(x) = n * p$
\newline\newline Standartabweichung:
Wo kommt das q her?

$\sigma^2 = V(x)= n * p * q$
\subsection{Poissionverteilung}
Mittelwert:

$\mu = n * p$
\newline\newline Warscheinlichkeit:

$P(x=k) = \frac{\mu^k} {k!} e^{-\mu}$
\newline\newline Warscheinlichkeitsdichte:

$f(x) = \frac{\mu^x} {x!} e^{-\mu}$
\newline\newline Erwartungswert:

$\mu = E(x) = n * p$
\newline\newline Standartabweichung:

$\sigma^2 = V(x)= n * p = \mu $
\subsection{Gleichverteilung}
Warscheinlichkeitsdichte:

$f(x) = \begin {cases}
  \frac{1} {b-a}, & \text {für } a\le x \le b \\
  0, & \text{sonst}
\end{cases} $
\newline\newline Erwartungswert:

$ \mu = E(x) = \frac { a + b } {2}$
\newline\newline Standartabweichung:

$\sigma^2 = V(x)= \frac {(a + b)^2 } {12} $
\subsection{Normalverteilung}

\subsubsection{Allgemein}
Warscheinlichkeitsdichte: 

$f(x) = \frac{1} {\sigma \sqrt{2\pi}} e^{(-\frac{1}{2}(\frac{x-\mu}{\sigma})^2)}$
\newline\newline Erwartungswert:
\newline Der Erwartugswert kann durch ableiten errechnet werden.

$f'(x) = -\frac{x-\mu} {\sigma^2} f(x)$
$\mu$ erhält man wenn man f'(x) = 0 setzt.
\newline\newline Standartabweichung:
\newline Die Standartabweichung erhält man durch 2 maliges ableiten.


$f''(x) = \frac{1}{\sigma^2} (\frac{1}{\sigma^2} (x-\mu)^2 - 1) f(x)$


$f''(x) = 0 $ liefert $ x = \mu \pm \sigma$

\subsubsection{Standardnormalverteilung}
Für die Standardnormalverteilung gilt:


$\mu = 0$


$\sigma = \pm 1$

\subsection{$\chi^2-Verteilung$}
\subsection{T-Verteilung}
\section{Carakteristische Funktion}
\section{Maximum-Likelihood-Methode (LM)}
\section{Konvidenz intervalle}
\section{Tests}
\subsection{T-Test}
\subsection{Biomial-Test}
\subsection{$\chi^2$-Test}

\end{document}
